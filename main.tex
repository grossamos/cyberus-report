%%%%%%%%%%%%%%%%%%%%%%%%%%%%%%%
%                             %
% AMOS ULTRA TEMPLATE 9000000 %
%                             %
%%%%%%%%%%%%%%%%%%%%%%%%%%%%%%%

\documentclass[a4,10pt]{article} 

\usepackage{opticameet3} 

\usepackage{amsmath,amssymb}
\usepackage[printonlyused,footnote]{acronym}
\usepackage{listings}
\usepackage[]{hyperref} %pdflatex

\definecolor{LinkColor}{rgb}{0,0,0.2}
\hypersetup{%
	colorlinks=true, % Aktivieren von farbigen Links im Dokument
	linkcolor=LinkColor, % Farbe festlegen
	citecolor=LinkColor,
	filecolor=LinkColor,
	menucolor=LinkColor,
	urlcolor=LinkColor,
	bookmarksnumbered=true % Überschriftsnummerierung im PDF Inhalt anzeigen.
}

\begin{document}

\title{Title}

\author{Amos Groß \normalfont{(\textit{mail@etud.univ-ubs.fr})}}

\section{Title One}

\subsection{Title One point one} 
Some cool text

\begin{thebibliography}{99} %% use BibTeX or add references manually

\bibitem{gentle2006} J. E. Gentle, \textit{Random number generation and Monte Carlo methods (2nd ed.)}, New York: Springer, p. 38, ISBN 0-387-00178-6

\end{thebibliography}

\section*{Acronymns}
\begin{acronym}[XXXXXXX]
    \acro{GNU}{GNU's not Unix!} 
\end{acronym}

\end{document}

